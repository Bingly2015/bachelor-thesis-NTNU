\chapter{Implementation}
\label{chap:implementation}
This has the description of how you actually went about implementing the project.  This should be focused on the interesting challenges and how those related to the project.

\todo{add more here. if you are reading this you can see that I am using todo as a way to indicate where the updates should be}


For code listing we have included the listings package so that you can easily include formatted code.  It does not have code highlighting but it retains the structure of the code.  For more documentation on listings on wikibooks \footnote{\url{https://en.wikibooks.org/wiki/LaTeX/Source_Code_Listings}}

\lstset{language=Python}
\begin{lstlisting}
import numpy as np
 
def incmatrix(genl1,genl2):
    m = len(genl1)
    n = len(genl2)
    M = None #to become the incidence matrix
    VT = np.zeros((n*m,1), int)  #dummy variable
 
    #compute the bitwise xor matrix
    M1 = bitxormatrix(genl1)
    M2 = np.triu(bitxormatrix(genl2),1) 
 
    for i in range(m-1):
        for j in range(i+1, m):
            [r,c] = np.where(M2 == M1[i,j])
            for k in range(len(r)):
                VT[(i)*n + r[k]] = 1;
                VT[(i)*n + c[k]] = 1;
                VT[(j)*n + r[k]] = 1;
                VT[(j)*n + c[k]] = 1;
 
                if M is None:
                    M = np.copy(VT)
                else:
                    M = np.concatenate((M, VT), 1)
 
                VT = np.zeros((n*m,1), int)
 
    return M
\end{lstlisting}



