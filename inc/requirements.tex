\chapter{Requirements}
\label{chap:requirements}

The title of the thesis should be set using the \verb+\thesistitle+
command, and the date of the thesis should be set using the
\verb+\thesisdate+ command. This makes the title and date appear in
the running header, like in this document.

\section{Page Layout}

The geometry of the page has been set using the \verb+\geometry+
command. 

\section{Fonts}

Due to limited \LaTeX\ support for the Georgia font, Charter has been
chosen instead. For mathematical formula, the Euler fonts are used,
since they blend more nicely with the Charter than the standard
\LaTeX\ fonts: 
$$
 f(x) = \int_0^x g(\tau)\,d\tau
$$

For inline math you can use $\backslash{}($ and $\backslash{})$ for example \( f(x)= \frac{x^2}{1+x^2} \).  
This also allows you to use $\slash$ and $\backslash$. You need to include the \{\} when you want the special
character to have other letters immediately after it.

\section{Sectioning Commands}

The standard \LaTeX\ sectioning commands are used for both numbered
and unnumbered sections. The top level is given by the \verb+\chapter+
command. This starts a new right page. The two lower levels are
obtained using the \verb+\section+ and \verb+\subsection+ commands.
The standard \LaTeX\ \verb+\subsubsection+ and \verb+\paragraph+
commands have been disabled since their use is not encouraged by the
thesis guidelines. When you use these they will not be given numbers.  
They still appear in the document with highlighting but not in the 
table of contents.

\subsection{The subsection}

This is an example of a subsection.

\subsubsection{The subsubsection}

This is an example of a subsubsection.

\paragraph{The paragraph}

This is an example of a paragraph with a heading.

\section{Floats (Figures and Tables)}
\label{sec:floats}

Figures are placed in the \texttt{figure} environment. An example is
shown in Figure~\ref{fig:examplegnuplot}. %notice the ~ in between figure and the \ref. it stops latex from splitting the number and word over a line.
You can make nicer graphs using gnuplot, for example see Figure~\ref{fig:examplegnuplot}.

Tables are placed in the \texttt{table} environment. An example is given in
Table~\ref{tab:example1}. Figures and tables float freely around in the
document in accordance with standard \LaTeX\ behavior.

\begin{figure}[htbp]  %t top, b bottom, p page | you can also use h to try to get the figure to appear at the current location
  \centering
  \includegraphics[width=.5\textwidth]{images/example_fig}
  \caption[An example figure.]{An example figure. If the caption is
    shorter than one line, it is centered. If it goes over more than
    one line, it is left and right justified. Furthermore, it is
    suggested that an alternative short caption is given in order to
    produce a good list of figures.}
  \label{fig:examplegnuplot}
\end{figure}

\begin{figure}[htbp]  %t top, b bottom, p page | you can also use h to try to get the figure to appear at the current location
  \centering
  \includegraphics[width=.5\textwidth]{images/kart_student}
  \caption[Map of NTNU Campuses]{The map shows the three main campuses of NTNU.}
  \label{fig:mapNTNU}
\end{figure}


\begin{figure}[htbp]  %t top, b bottom, p page | you can also use h to try to get the figure to appear at the current location
  \centering
  \includegraphics[width=.5\textwidth]{images/stylus}
  \caption[Stylus]{A standard stylus}
  \label{fig:stylus}
\end{figure}



\begin{table}[tbp]
  \centering
  \begin{tabular}{c|c|c}
    Age  & IQ  & \\ 
    \hline
    10   & 100 \\
    20   & 100 \\
    30   & 150 \\
    40   & 100 \\
    50   & 100
  \end{tabular}
  \caption{An example table.}
  \label{tab:example1}
\end{table}

\begin{figure}[tbp]
  \centering
  \csvautobooktabular{images/ageiq.csv}
  \caption{An example table using simplecsv.}
  \label{tab:examplecsv}
\end{figure}

The captions are placed \emph{below} both for the figures and the
tables. The caption is set in 9pt. If the caption is shorter than one
line, it is centered.

\section{Quotes}
\label{sec:Quotes} % this allows you to refer to this section number using \ref{sec:Quotes}

Quotes are inserted using the standard \LaTeX\ \texttt{quote}
environment. The environment has been changed so that a 9pt font is
used:

\begin{quote}
  ``And I looked, and, behold, a whirlwind came out of the north, a
  great cloud, and a fire infolding itself, and a brightness was about
  it, and out of the midst thereof as the colour of amber, out of the
  midst of the fire. Also out of the midst thereof came the likeness
  of four living creatures.''
\end{quote}

\section{Lists}
\label{sec:lists}

Point lists and enumerated lists are made by using the standard
\texttt{itemize} and \texttt{enumerate} environments, respectively.
The spacing is going to be changed in accordance with the specification. For
\texttt{itemize}, the results look like this:
\begin{itemize}
	\item First item.
	\item Second item. Here I will put some long text, just to illustrate.
	  Here I will put some long text, just to illustrate. Here I will put
	  some long text, just to illustrate. Here I will put some long text,
	  just to illustrate.
	\item Third item also has subitems:
	  \begin{itemize}
		  \item First subitem.
		  \item Second subitem.
		  \item Third subitem.
			  \end{itemize}
\end{itemize}
and for \texttt{enumerate} like this:
\begin{enumerate}
	\item First item.
	\item Second item. Here I will put some long text, just to illustrate.
	  Here I will put some long text, just to illustrate. Here I will put
	  some long text, just to illustrate. Here I will put some long text,
	  just to illustrate.
	\item Third item also has subitems:
	  \begin{enumerate}
		  \item First subitem.
		  \item Second subitem.
		  \item Third subitem.	  
			\begin{enumerate}
		  \item First subitem.
		  \item Second subitem.
		  \item Third subitem.
	  \end{enumerate}
	  \end{enumerate}
\end{enumerate}

You may also want to use descriptive lists
\begin{description}
	\item[First] the first item.
	\item[Second] the second item. Here I will put some long text, just to illustrate.
	  Here I will put some long text, just to illustrate. Here I will put
	  some long text, just to illustrate. Here I will put some long text,
	  just to illustrate.
	\item [What now] the third item also has subitems:
	  \begin{enumerate}
		  \item First subitem.
		  \item Second subitem.
		  \item Third subitem.
	  \end{enumerate}
\end{description}


\section{Bibliographic References}

You should cite articles~\cite{Askvall1985}, books~\cite{Card1983},
anthologies~\cite{Lancaster1985} and web publications~\cite{Meldon1997}
like this.


A particular bibliography style file for NTNU named
\texttt{ntnubachelorthesis.bst} has been developed based upon the
standard Bib\TeX\ \texttt{unsrt} style.
